\documentclass{article}     
\usepackage[utf8]{inputenc}	
\usepackage[norsk]{babel}
\usepackage{amsmath,amsfonts,amssymb,amsthm,mathtools} 
\usepackage{multirow}
\usepackage{graphicx}		% Graphics package
\graphicspath{ {./images/} }

\usepackage[center]{caption}
\usepackage{subcaption}
\usepackage{wrapfig}  		% Wrap text around figures	
\usepackage{float} 			% To place figures correctly
\usepackage{listings} 		% To include code-snippets
\usepackage{color}
\usepackage{hyperref} 		
\usepackage{siunitx}  		% Write units easily \SI{value}
\usepackage[utf8]{inputenc}
\lstset{
  basicstyle=\footnotesize\ttfamily,
  identifierstyle=\bfseries\color{green!40!black},
  commentstyle=\itshape\color{purple!40!black},
  keywordstyle=\color{blue},
  stringstyle=\color{orange},
}
\usepackage{changepage}
\usepackage{pdfpages}	
\usepackage{lscape} 
\usepackage[authoryear,square]{natbib} 
% References/bibliography package
\usepackage{footnote}
\usepackage{appendix} 
% Appendix + add to table of contents
\usepackage{sidecap}
% Table packages
\usepackage{tabularx}
\makesavenoteenv{tabular}
\makesavenoteenv{table}
\usepackage{array}
\usepackage{booktabs}
\usepackage{rotating}
\newcommand*\rot{\rotatebox{90}}
\newcommand{\E}{\mathop{\mathbb{E}}}


\title{Homeexam 3; STA-3001}
\author{Isak Edvardsen }
\date{March 2020}

\begin{document}

\maketitle

\section{1: EM, MCEM, DA}

\vspace{0.5cm}

Assume data from $n=20$ street lamps, each having two light bulbs.\\

If we assume that the failure times of individual light bulbs are independent and exponentially distributed with mean $1/\lambda$ then the failure times of the street lamps is gamma distributed (with $\alpha = 2$):

\begin{equation}
    f(x|\lambda) = \lambda^2 x e^{-x\lambda}, \hspace{.5cm} x>0, \hspace{.5cm} \lambda >0
\end{equation} 
\\

\noindent Assume a random sample of n observations of failure times of street lamps, some of these are censored and we observe $\{ X_i\}_{i=1}^{n+m}$. \\

\noindent We'll suppose $\{ X_i \}_{i=1}^m$ are observed, but $X_i = c$ for $m + 1 \leq i \leq n$ (only know that they had not failed yet at a certain time $c$).\\

The likelihood is

\begin{equation}
    L(X;\lambda) = \bigg( \prod_{i=1}^{m} f(x_i|\lambda) \bigg) \cdot \bigg( \prod_{i=m+1}^{n} f(x_i|\lambda) \bigg)
\end{equation}


The observations $\{ X_i \}_{i=m+1}^n$ can be treated as if they were missing.


Define the complete observations $Z = \{ X_i \}_{i=m+1}^{n}$, hence Z contains the unobserved failure times. \\

The likelihood of $Y = (X,Z)$ (independent) is:

\begin{equation}
    L(Y;\lambda) = \bigg( \prod_{i=1}^{m} f(x_i|\lambda) \bigg) \cdot \bigg( \prod_{i=m+1}^{n} f(z_i|\lambda) \bigg)
\end{equation}

\newpage 

the completed log-likelihood is:

\begin{align}
    logL(Y;\lambda) &= log \bigg\{ \prod_{i=1}^{m} f(x_i|\lambda)\cdot \prod_{i=m+1}^{n} f(z_i|\lambda) \bigg\} \\ 
    &= \sum_{i=1}^{m} logf(x_i|\lambda) +  \sum_{i=m+1}^{n} logf(z_i|\lambda) \nonumber \\
    &= \sum_{i=1}^{m} log\big(\lambda^2 x_i e^{-x_i\lambda}\big) + \sum_{i=1}^{n} log\big(\lambda^2 z_i e^{-z_i\lambda}\big) \nonumber \\
    &= \sum_{i=1}^{m}\big(2log(\lambda) + log(x_i)+x_i\lambda \big) + \sum_{i=1+m}^{n}\big( 2log(\lambda) + log(z_i) + z_i \lambda\big) \nonumber \\
    &= 4log(\lambda) + \sum_{i=1}^{m}\big(log(x_i) + x_i \lambda\big) + \sum_{i=1+m}^{n}\big(log(z_i) + z_i \lambda\big)
\end{align}


\begin{align}
   \E (Z_i | \boldsymbol x, \lambda) &= \E f(z|\boldsymbol x, \lambda) (\lambda|\boldsymbol x, z) \nonumber \\
    &= \E 2
\end{align}




\end{document}
